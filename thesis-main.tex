\documentclass[12pt,a4paper,openright,twoside]{book}
\usepackage[english,italian]{babel}
\usepackage[utf8]{inputenc}
\usepackage{disi-thesis}
\usepackage{notes}
\usepackage{shortcuts}
\usepackage{acronym}
\usepackage{code-lstlistings}

\school{\unibo}
\programme{Corso di Laurea in Ingegneria e Scienze Informatiche}
\title{Studio e sviluppo di un ambiente di
emulazione per dispositivi IoT con
interfaccia Web of Things}
\author{D'Antino Matilde}
\date{\today}
\subject{Embedded Systems and Internet Of Things}
\supervisor{Prof. Ricci Alessandro}
\cosupervisor{Dott. Burattini Samuele}
\cosupervisor{Dott. Giulianelli Andrea}
\session{IV}
\academicyear{2024-2025}


\mainlinespacing{1.241}
\selectlanguage{italian}

\begin{document}

\frontmatter\frontispiece

\begin{myabstract}
    
\end{myabstract}

\begin{dedication} 
\end{dedication}


\tableofcontents

\lstlistoflistings
\listoffigures

\mainmatter

%----------------------------------------------------------------------------------------
\chapter{Introduzione}
\label{chap:introduzione}

%----------------------------------------------------------------------------------------


%----------------------------------------------------------------------------------------
\chapter{Contesto}
\label{chap:contesto}
\section{Contesto}
\section{Web Of Things}

%----------------------------------------------------------------------------------------

%----------------------------------------------------------------------------------------
\chapter{Analisi}
\label{chap:analisi}

\section{Analisi dei requisiti}
\section{Modello del dominio}

\
%----------------------------------------------------------------------------------------

%----------------------------------------------------------------------------------------
\chapter{Design}
\label{chap:design}

\section{Struttura}

\section{Linguaggi e Tecnologie utilizzate}
\subsection{Backend}
\subsection{Frontend}

\
%----------------------------------------------------------------------------------------

%----------------------------------------------------------------------------------------
\chapter{Caso di studio}
\label{chap:caso di studio}

%----------------------------------------------------------------------------------------

%----------------------------------------------------------------------------------------
\chapter{Web Containers}
\label{chap:web containers}

%----------------------------------------------------------------------------------------


%----------------------------------------------------------------------------------------
% BIBLIOGRAPHY
%----------------------------------------------------------------------------------------

\backmatter

\bibliographystyle{alpha}
\bibliography{bibliography}

\begin{conclusions}
\end{conclusions}

\begin{acknowledgements}
\end{acknowledgements}

\end{document}
